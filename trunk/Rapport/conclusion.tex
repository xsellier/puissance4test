
\section{conclusion}


La simplicité du support Puissance 4 nous a permis de ne pas nous focaliser sur des problèmes de réalisation. Nous avons ainsi pu prendre du temps pour développer une stratégie de test ou du moins une série de tests et tenter d'aborder les différents outils et phases vus en cours. Nous nous sommes d'efforcés de couvrir une variété de tests aussi large que possible. Ainsi, on peut observer des tests classiques que des tests unitaires réalisés à l'aide de JUnit ou des parcours de chemins du programme basés sur une analyse statique. Le nombre de participants au projet n'a pas toujours facilité notre organisation puisqu'il était difficile d'organiser des réunions au travers de nos examens de Janvier. Nous estimons cependant que le code a atteint une certaine maturité et que le rapport regroupe une quantité convenable d'informations au regard du travail effectué.

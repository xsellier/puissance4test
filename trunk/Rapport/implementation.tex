% -*- mode: latex; coding: latin-1-unix -*- %

\section{Implementation}

\subsection{DataStructure}
\subsubsection{Matrice}
\begin{verbatim}
   private int[][] matrix;
   private int height;
   private int width;
\end{verbatim}
Nous avons choisi d'utiliser une matrice pour mod�liser une grille de
Puissance 4. Ce sera un tableau d'entiers. Afin de faire moins de
calculs nous avons choisi de stocker la hauteur ainsi que la largeur
de la matrice dans deux variables distinctes \texttt{height} respectivement \texttt{width}.

\begin{verbatim}
   public DataStructure(int height, int width);
\end{verbatim}
Le constructeur va prendre deux entiers en param�tres. Ce constructeur
va v�rifier si ces entiers sont valides, autrement dit si ils ne sont
pas n�gatifs ou nuls.

\begin{verbatim}
   public int getHeight();
   public int getWidth();
   public int getValue(int i, int j);
   public boolean setValue(int i, int j, int color)
\end{verbatim}
Nous avons pr�f�r� mettre nos variables (\texttt{matrix}, \texttt{height} et \texttt{width}) en
private pour �viter toutes modifications inattendues de notre
matrice. d'o� l'existence de ces accesseurs. Nous n'avons pas fait
d'accesseur direct � la matrice, autrement dit qui retournerait notre
matrice, toujours dans le soucis de modifications inattendues.

La m�thode \texttt{setValue} va modifier notre matrice en respectant la
contrainte disant que \texttt{i} et \texttt{j} doivent etre compris entre 0 et respectivement
\texttt{height} et \texttt{width}.

Nous n'avons pas mis de restriction sur color, etant donn� que la
structure de donn�es ne g�re pas les r�gles du jeu, elle ne sait pas
de quoi il s'agit exactement.

Cette m�thode retourne un bool�en qui renvoie \texttt{true}
respectivement \texttt{false} si la modification a pu �tre apport� ou pas.

\begin{verbatim}
   public void reset_matrix()
   public void print()
\end{verbatim}
La m�thode \texttt{reset\_matrix} va, comme son nom l'indique, faire un
simple reset de la matrice.

La m�thode \texttt{print} va elle afficher la matrice.


\subsubsection{Tests de la matrice}

\subsection{Player}
\subsubsection{Impl�mentation}

\subsubsection{Jeux de test}

\subsection{HumanPlayer}
\subsubsection{Impl�mentation}

\subsubsection{Jeux de test}

\subsection{CpuPlayer}
\subsubsection{Impl�mentation}

\subsubsection{Jeux de test}

\subsection{Rules}
\subsubsection{Impl�mentation}

\subsubsection{Jeux de test}

\subsection{GameEngine}
\subsubsection{Impl�mentation}

\subsubsection{Jeux de test}

\subsection{Main et Menu}
\subsubsection{Impl�mentation}

\subsubsection{Jeux de test}

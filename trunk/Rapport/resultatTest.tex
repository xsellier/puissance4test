\section{Tests syst�me}


\section{Tests d'int�gration}


\section{Tests unitaires}

\subsection{DataStructureTest}

La plupart des bugs identifi�s seront li�s aux valeurs des positions de la grille.\newline
Particuli�rement, une affectation d'une couleur � une position \textbf{hors limite} est source 
de bugs.\newline
Le fait d'affecter une couleur quelconque � la position \textbf{(0,9)} d'une grille (8,9) est typiquement le type de bugs qu'on aura � traiter.\newline


\subsubsection{Test 1}

\begin{verbatim}
		public void testInvalidSetValues() {
				assertFalse(matrix.setValue(10, 1, 0));
				assertFalse(matrix.setValue(1, 10, 1));
				assertFalse(matrix.setValue(10, 10, 2));
		}
\end{verbatim}
   
\subsubsection{R�sultat Test 1}

Ce test montre bien qu'il n'est pas possible d'affecter des couleurs \textbf{hors des limites} 
� la grille \textbf{matrix} (6,7) consid�r�e.\newline

\subsubsection{Test 2}

\begin{verbatim}
		public void testNegativeMatrix() {
				int resultNH = negative_matrix.getHeight();
				assertEquals("la hauteur d'une negative_matrix n'est pas  6", resultNH, 6);

				int resultNW = negative_matrix.getWidth();
				assertEquals("la largeur d'une negative_matrix n'est pas  6", resultNW, 7);
		}    
\end{verbatim}

\subsubsection{R�sultat Test 2}

Ce test montre que les param�tres (height , width) d'une grille "n�gative" n'ont 
de valeur que respectivement 6(hauteur) et 7(largeur).  \newline
Ce m�me type de test sera experiment� sur une grille "zero" et donnera le m�me r�sultat.\newline

\subsubsection{Test 3}

Tester \textbf{aux limites} la possiblit� d'affecter des couleurs � une grille (6,7).\newline
Ce test peut �tre g�n�ralis� � des grilles (i,j) avec i>0 et j>0.\newline  

\subsubsection{R�sultat Test 3}

En effet, l'affectation des couleurs dans le Domaine des positions [0...42] r�ussit avec succ�s.
Mais, pour les valeurs -1 et 43 le test montre une impossibilit� d'affectation.\newline
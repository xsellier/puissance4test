% -*- mode: latex; coding: latin-1-unix -*- %

\subsection{Tests Unitaires}

\subsubsection{DataStructureTest}

La plupart des bugs identifi�s seront li�s aux valeurs des positions de la grille.\newline
Particuli�rement, une affectation d'une couleur � une position \textbf{hors limite} est source 
de bugs.\newline
Le fait d'affecter une couleur quelconque � la position \textbf{(0,9)} d'une grille (8,9) est typiquement le type de bugs qu'on aura � traiter.

\paragraph{Test 1} - R�sultats

\begin{verbatim}
    public void testInvalidSetValues() {
        assertFalse(matrix.setValue(10, 1, 0));
        assertFalse(matrix.setValue(1, 10, 1));
        assertFalse(matrix.setValue(10, 10, 2));
    }
\end{verbatim}
   
Ce test montre bien qu'il n'est pas possible d'affecter des couleurs \textbf{hors des limites} 
� la grille \textbf{matrix} (6,7) consid�r�e.

\paragraph{Test 2} - R�sultats

\begin{verbatim}
    public void testNegativeMatrix() {
        int resultNH = negative_matrix.getHeight();
        assertEquals("la hauteur d'une negative_matrix n'est pas  6", resultNH, 6);

        int resultNW = negative_matrix.getWidth();
        assertEquals("la largeur d'une negative_matrix n'est pas  6", resultNW, 7);
   }    
\end{verbatim}

Ce test montre que les param�tres (height , width) d'une grille "n�gative" n'ont 
de valeur que respectivement 6(hauteur) et 7(largeur).  \newline
Ce m�me type de test sera experiment� sur une grille "zero" et donnera le m�me r�sultat.

\paragraph{Test 3} - R�sultats

\bigskip
Tester \textbf{aux limites} la possiblit� d'affecter des couleurs � une grille (6,7).\newline
Ce test peut �tre g�n�ralis� � des grilles (i,j) avec i>0 et j>0.\newline  

En effet, l'affectation des couleurs dans le Domaine des positions [0...42] r�ussit avec succ�s.
Mais, pour les valeurs -1 et 43 le test montre une impossibilit� d'affectation.\newline

\subsubsection{IAfourInARow}

Ces tests nous ont permis de r�v�ler de nombreux bugs li�e aux comportements des IA. Tout d'abord sur les strat�gies de \texttt{playable[]} qui n'�tait pas toujours juste par rapport � l'�tat actuel de la grille,et aussi le fa�on de jouer des IA ne correspondait pas toujours � ce qui �tait attendut dut au fait des priorit�s des \texttt{playable}.

Pour chaque �tat de la grille ils nous �taient possible a l'aide de l'algorithme de connaitre l'�tat de \texttt{playable[]} et le coup que doit jouer l'IA ainsi nous avons corriger les moteurs d'IA pour qu'ils nous donnes exactement les r�sultats escompt�s dans les cas typiques ci-dessus.

%%%%%%%cfa%%%%%%

L'ensemble des tests ci-dessus permmettent de couvrir int�gralement les arr�tes de l'Automate de flow de contr�le de l'IA en mode difficile ( et donc de l'IA facil qui est une sous partie), nous sommes donc dans le cas d'une couverture TER2, la complexit� du CFA ne nous permet pas d'obtenit des TER sup�rieur sans une d�multiplication des tests.


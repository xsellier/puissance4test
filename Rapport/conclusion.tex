% -*- mode: latex; coding: latin-1-unix -*- %
\section{conclusion}


La simplicit� du support Puissance 4 nous a permis de ne pas nous focaliser sur des probl�mes de r�alisation. Nous avons ainsi pu prendre du temps pour d�velopper une strat�gie de test ou du moins une s�rie de tests et tenter d'aborder les diff�rents outils et phases vus en cours. Nous nous sommes d'efforc�s de couvrir une vari�t� de tests aussi large que possible. Ainsi, on peut observer des tests classiques que des tests unitaires r�alis�s � l'aide de JUnit ou des parcours de chemins du programme bas�s sur une analyse statique. Le nombre de participants au projet n'a pas toujours facilit� notre organisation puisqu'il �tait difficile d'organiser des r�unions au travers de nos examens de Janvier. Nous estimons cependant que le code a atteint une certaine maturit� et que le rapport regroupe une quantit� convenable d'informations au regard du travail effectu�.
